\documentclass{article}
\usepackage[utf8]{inputenc}
\usepackage[spanish]{babel}
\usepackage{listings}
\usepackage{graphicx}
\graphicspath{ {images/} }
\usepackage{cite}

\begin{document}

\begin{titlepage}
    \begin{center}
        \vspace*{1cm}
            
        \Huge
        \textbf{Parcial 1 - Informe}
            
        \vspace{0.5cm}
        \LARGE
        Informática II
            
        \vspace{1.5cm}
            
        \textbf{Ariel Martínez}
            
        \vfill
            
        \vspace{0.8cm}
            
        \Large
        Despartamento de Ingeniería Electrónica y Telecomunicaciones\\
        Universidad de Antioquia\\
        Medellín\\
        Abril de 2021
            
    \end{center}
\end{titlepage}

\tableofcontents
\newpage
\section{Análisis del problema y consideraciones para la alternativa de solución propuesta}\label{analisis}

\subsection{Problema}

El problema planetado para el parcial 1, consiste en la impleentación de un display LED en un arreglo de 8 x 8, utilizando el circuito integrado (IC) 74HC595 y el mínimo posible de pines de salida de un dispositivo Arduino.

El programa a implementar deberá permitir que un usuario ingrese a través de la consola serial un número arbitrario de patrones, los cuales se presentarán de forma secuencial en el display. Adicionalmente, el programa deberá permitir la verificación del funcionamiento del display.

\subsection{Solución propuesta}

El IC 74HC595 tiene una entrada serial y dos relojes. A través de la entrada serial se puede ingresar secuencias de 8 bits al IC, el cual  tiene una salida paralela de ocho bits, que se puede utilizar para activar una línea de LED.

La última salida de este IC, está destinada para conectar en serie otros IC. Por lo tanto, es posible conectar la salida serial de un IC a la entrada serial de otro, permitiendo multiplicar la salida en paralelo. De esta forma, es posible conectar ocho IC en serie utilzindo un solo pin de salida del Arduino. 
Los relojes no requieren ser puestos en serie, sino que se conectan a un mismo bus en paralelo en cada IC. Así, solamente se utilizan sólamente tres pines de salida del Arduino.

Las funcionalidades requeridas para el programa se realizarán de dos formas. Para la verificación del funcionamiento del display, se enviará una sóla secuencia de 64 bits en HIGH. Sin embargo, la presentación de patrones secuenciales arbitrarios requiere el uso de memoria dinámica, debido a que no se conoce de antemano cuántos patrones ingresará el usuario. Para esto se codificarán funciones específicas que permitan asignar sólo la memoria necesaria.

\section{Tareas definidas en el desarrollo del algoritmo}

Esquema que describa esto

\section{Algoritmo implementado}

Exportar algoritmo

\section{Problmeas de desarrallo presentados}

¿Problemas?

\section{Evolución del algoritmo y consideraciones para la implementación}



\section{Sección de contenido} \label{contenido}
Esta sección es para agregar toda la información correspondiente con código, citas, etc.
\subsection{Citación}
Vamos a citar por ejemplo un artículo de \textbf{Albert Einstein} \cite{einstein}.
También es posible citar libros \cite{dirac} o documentos en línea \cite{knuthwebsite}.\\\\
Revisar en la última sección el formato de las referencias en IEEE.

\subsection{Incluir código en el documento}
%
A continuación, se presenta el código \ref{codigo_ejemplo}, que nos permite incluir en el informe partes de programa que requieran una explicación adicional.
\begin{lstlisting}[language=C++, label=codigo_ejemplo]
// Programa desarrollado, compilado y ejecutado en https://www.onlinegdb.com

#include <iostream>

/*
 * Esto es un comentario de varias lineas
 */

// Comentario de una sola linea

#define N 10

using namespace std;

int main()
{
    
    for( int i = 0 ; i < N ; i++ ){
        
        if( !(i % 2) )
            cout << " El valor de i es -> " << i << endl;
    }
    
    return 0;
}

//Resultado programa

/*
El valor de i es -> 0
El valor de i es -> 2
El valor de i es -> 4
El valor de i es -> 6
El valor de i es -> 8
*/
\end{lstlisting}
En la sección \ref{imagenes}, se presentará como añadir ilustraciones al texto.

\section{Inclusión de imágenes} \label{imagenes}

En la Figura (\ref{fig:cpplogo}), se presenta el logo de C++ contenido en la carpeta images.

\begin{figure}[h]
\includegraphics[width=4cm]{cpplogo.png}
\centering
\caption{Logo de C++}
\label{fig:cpplogo}
\end{figure}

Las secciones (\ref{analisis}), (\ref{contenido}) y (\ref{imagenes}) dependen del estilo del documento.

\bibliographystyle{IEEEtran}
\bibliography{references}

\end{document}

